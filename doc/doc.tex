\documentclass{article}

\usepackage[utf8]{inputenc}
\usepackage{polski}
\usepackage{amsmath}
\usepackage{amsfonts}
\usepackage{indentfirst}

\begin{document}

\section{Dystanse}

Można badać korelacje między różnymi dystansami, żeby się czegoś dowiedzieć.

\subsection{Konkretne funkcje}

\subsubsection{Jaccard}

\subsubsection{Cosine}

\subsubsection{Chord}

Odległość Euklidesowa po zrzutowaniu na sferę jednostkową.
\[
    ||A - B||^2
    = (A-B)(A-B)
    = ||A||^2 + ||B||^2 - 2 (A \cdot B)
    = 2 (1 - \cos(A, B))
\]
Zatem z dokładnością do stałej, jest pierwiastkiem z odległości Cosine.

\subsubsection{SimRank}

\subsubsection{RoleSim}

\subsubsection{Najkrótsza ścieżka w grafie dwudzielnym}

\section{Wizualizacja}

Aby ocenić jakość wizualizacji, można zbadać korelację między odległością w
embeddingu a normalną.

\subsection{Metody}

\subsubsection{TSNE}

\subsubsection{MDS}

\subsubsection{PCA}

\subsubsection{UMAP}

Na razie nie.

\subsubsection{LDA}

Na razie nie.

\section{Community Detection}

\section{Klastrowanie}

\section{Ocena jakości klastrów}

\subsection{Różne współczynniki}

\subsubsection{Silhouette}

Dla każdego punktu ocenia jak dobrze został przydzielony do swojego klastra
liczbą z przedziału $[0,1]$.

\paragraph{Silhouette coefficient}

\[ SC = \max_k \tilde{s}(k) \]

Maksymalny średnia wartość Silhouette. Używana czasami do określenia jak
dobrze dana metoda klastrowania odnajduje klastry.

\subsubsection{Davies-Bouldin index}

\subsubsection{Dunn index}

\subsubsection{Calinski-Harabasz index}

\subsection{Walidacja lingwistyczna}

\section{Identyfikacja grup}

Zamiast podziału na klastry można próbować identyfikować podobne grupy.

\end{document}
